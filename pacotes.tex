% ---
% Pacotes básicos 
% ---	
\usepackage[T1]{fontenc}		% Selecao de codigos de fonte.
\usepackage[utf8]{inputenc}		% Codificacao do documento (conversão automática dos acentos)
\usepackage{indentfirst}		% Indenta o primeiro parágrafo de cada seção.
\usepackage{color}				% Controle das cores
\usepackage{graphicx}			% Inclusão de gráficos
\usepackage{titlesec} % Possibilita a formatação de capítulos, etc
\usepackage{microtype} 			% para melhorias de justificação
\usepackage{IEL} % Inclusão do modelo IEL
\usepackage{helvet} % Fonte Helvética
\usepackage[table,xcdraw]{xcolor} % Cores em tabelas
% ---
		
% ---
% Pacotes adicionais, usados apenas no âmbito do Modelo Canônico do abnteX2
% ---
\usepackage{lipsum}				% para geração de dummy text
% ---

% ---
% Pacotes de citações
% ---
\usepackage[brazilian,hyperpageref]{backref}	 % Paginas com as citações na bibl
\usepackage[alf, versalete,
abnt-emphasize = bf, % destaca o titulo em negrito;
abnt-etal-list = 3, % trabalhos com mais de 3 autores recebem et al.,;
abnt-etal-text = it, % escreve o et al., em italico;
abnt-and-type = &, % usa o carater '&' no lugar de 'e' para mais de um autor;
abnt-last-names = abnt] % trata sobrenomes 'estritamente' conforme a ABNT; e
{abntex2cite}	% Citações padrão ABNT
% ---

% --- 
% CONFIGURAÇÕES DE PACOTES
% --- 
% ---
% Configurações do pacote backref
% Usado sem a opção hyperpageref de backref
\renewcommand{\backrefpagesname}{Citado na(s) página(s):~}
% Texto padrão antes do número das pápginas
\renewcommand{\backref}{}
% Define os textos da citação
\renewcommand*{\backrefalt}[4]{
	\ifcase #1 %
		Nenhuma citação no texto.%
	\or
		Citado na página #2.%
	\else
		Citado #1 vezes nas páginas #2.%
	\fi}%
% ---